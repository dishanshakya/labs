\documentclass{article}
\usepackage[a4paper]{geometry}
\begin{document}
\section*{Objective}
{\bfseries To draw a triangle and apply rotation and reflection transformations using C/C++}

\section*{Theory}
Transformation refers to changing the shape, size and orientation of an object. These are accomplished with \emph {geometric transformations} that alter the coordinate descriptions of objects. The basic geometric transformations are translation, rotation, reflection and scaling.
\subsection*{Reflection}
Reflection is a transformation that produces a mirror image of an object. The mirror image for a two-dimensional reflection is generated relative to an axis of reflection by rotating the object $180^{\circ}$ about the reflection axis.
\par Reflection on the line $y=0$, the x axis, is accomplished as
\begin{displaymath}
	(x',y') = (x, -y)
\end{displaymath}
\par Reflection on the line $x=0$, the y axis, is accomplished as
\begin{displaymath}
	(x',y') = (-x, y)
\end{displaymath}
\par Reflection on the line $y=x$ is accomplished as
\begin{displaymath}
	(x',y') = (y,x)
\end{displaymath}
\subsection*{Rotation}
A two-dimensional rotation is applied to an object by repositioning it along a circular path in the xy plane. To generate a rotation, we specify a rotation angle $\theta$ and the position $(x_r, y_r)$ of the rotation point about which the object is to rotated.
\par Rotation by an angle $\theta$ about the origin is accomplished as
\begin{eqnarray*}
	x' = x \cos \theta - y \sin \theta \\
	y' = x \sin \theta + y \cos \theta \\
\end{eqnarray*}
\par Rotation by angle $\theta$ about an arbitary point $(x_r, y_r)$ is accomplished as
\begin{eqnarray*}
	x' = x_r + (x - x_r)\cos \theta - (y-y_r) \sin \theta \\
	y' = y_r + (x - x_r)\sin \theta + (y - y_r) \cos \theta \\
\end{eqnarray*}
\end{document}
